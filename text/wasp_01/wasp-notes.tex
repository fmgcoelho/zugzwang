% !TEX program=pdflatex

\documentclass[x11names]{article}

\usepackage[style=authoryear]{biblatex}
\addbibresource{zugzwang.bib}

\usepackage{multirow}

% LNAI stuff

% TLP stuff

\usepackage{times}
% \usepackage{soul} % to strikeout, highline, etc.
\usepackage{url}
\usepackage[hidelinks]{hyperref}
\usepackage{graphicx}
\usepackage{booktabs}
\usepackage{algorithm}
\usepackage{algorithmic}
\urlstyle{same}
\usepackage[%textsize=tiny,
    prependcaption]{todonotes}
%

% OUR PACKAGES AND MACROS


\usepackage{tikz}
\tikzset{
    boxed/.style={draw, rectangle, rounded corners=4pt, fill=gray!10,inner sep=1em, anchor=center},
    event/.style={},
    smodel/.style={fill=gray!25},
    tchoice/.style={draw, circle},
    indep/.style={},
    proptc/.style = {-latex, dashed},
    propsm/.style = {-latex, thick},
    doubt/.style = {gray} }
\usetikzlibrary{calc, positioning, patterns, perspective}
%
\usepackage{hyperref}
\hypersetup{
    colorlinks=true,
    linkcolor=blue,
    citecolor=blue,
    urlcolor=blue, }
%
\usepackage{commath}

%\usepackage{ntheorem}
%\newtheorem{assumption}{Assumption}
\newtheorem{assumption}{Assumption}
\usepackage{amssymb}
\usepackage[normalem]{ulem}
% \usepackage[euler-digits,euler-hat-accent]{eulervm}
\usepackage[nice]{nicefrac}
\usepackage{stmaryrd}
\usepackage[smaller]{acronym}
\usepackage{multicol}
\usepackage{multirow}
\usepackage{cleveref}
\crefname{example}{ex.}{exs.}
\crefname{proposition}{prop.}{props.}
\crefname{assumption}{assumption}{assumptions}
\usepackage{hyphenat}
\hyphenation{
mo-dels mi-ti-ga-ted mi-ni-mal ex-am-ple spe-ci-fi-ca-tion
}
% Environments

\newtheorem{example}{Example}
\newtheorem{definition}{Definition}
\newtheorem{proposition}{Proposition}

% Commands

\def\tight{%
  \itemsep 0pt plus 1pt
  \parskip 0pt plus 1pt}

\newcommand{\ie}{\emph{i.e.}}
\newcommand{\eg}{\emph{e.g.}}
\newcommand{\eat}[1]{}
\newcommand{\at}[1]{\ensuremath{\!\del{#1}}}        %   argument
%
%   Notation
%
%   special sets
%
\newcommand{\cla}[1]{\ensuremath{{\mathcal{#1}}}}        %   class of something  
\newcommand{\clx}[1]{\ensuremath{{\mathbb{#1}}}}
%
%   connectives in logic programs
%
\newcommand{\conj}{\ensuremath{\wedge}} \newcommand{\disj}{\ensuremath{\vee}}
\newcommand{\clause}{\ensuremath{\leftarrow}}
\DeclareMathOperator{\naf}{\sim\!}
% \newcommand{\naf}{\ensuremath{\sim\!\!}}
\newcommand{\co}[1]{\ensuremath{\overline{#1}}}     %   complement
%\renewcommand{\complement}{\ensuremath{\complement \complement}}     %   complement
%
%   often used sets
%
\newcommand{\ATOMSset}{\ensuremath{\cla{A}}}
\newcommand{\Rset}{\ensuremath{\mathbb{R}}}
\newcommand{\LITERALSset}{\ensuremath{\cla{L}}}
\newcommand{\PATOMset}{\ensuremath{\ATOMSset_{\cla{P}}}}
\newcommand{\WATOMset}{\ensuremath{\ATOMSset_{\cla{W}}}}
\newcommand{\FACTSset}{\ensuremath{\cla{F}}}
\newcommand{\PROBFset}{\ensuremath{\cla{P}}}
\newcommand{\WEIGHTFset}{\ensuremath{\cla{W}}}
\newcommand{\RULESset}{\ensuremath{\cla{R}}}
\newcommand{\TCHOICEset}{\ensuremath{\cla{T}}}
\newcommand{\MODELset}{\ensuremath{\cla{M}}}
\newcommand{\EVENTSset}{\ensuremath{\cla{E}}}
\newcommand{\CONSISTset}{\ensuremath{\cla{C}}}
% \newcommand{\FRUITFUL}{\ensuremath{P_{\mathrm{fuitful}}}}
\newcommand{\SBF}{\ensuremath{P_{\sbf}}}
%
%   often used functions
%
%   error (loss)
\newcommand{\err}[1]{\ensuremath{\mathrm{err}\at{#1}}}
%
%   probability 
\newcommand{\prfunc}{\ensuremath{\mathrm{P}}}       
%   P(X = x)
\newcommand{\pr}[1]{\ensuremath{\prfunc\at{#1}}}    
%   P_X(x)
\newcommand{\prd}[1]{\ensuremath{\prfunc_{#1}}}     
\newcommand{\prT}{\prd{\TCHOICEset}}
\newcommand{\prM}{\prd{\MODELset}}
\newcommand{\prE}{\prd{\EVENTSset}}
\newcommand{\prC}{\prd{\cla{C}}}
%
%   weights 
\newcommand{\wgtfunc}{\ensuremath{\omega}}       
%   P(X = x)
\newcommand{\wgt}[1]{\ensuremath{\wgtfunc\at{#1}}}    
%   P_X(x)
\newcommand{\wgtd}[1]{\ensuremath{\wgtfunc_{#1}}}     
\newcommand{\wgtT}{\wgtd{\TCHOICEset}}
\newcommand{\wgtM}{\wgtd{\MODELset}}
\newcommand{\wgtE}{\wgtd{\EVENTSset}}
\newcommand{\wgtC}{\wgtd{\cla{R}}}
\newcommand{\wgte}[1]{\ensuremath{\wgtE\at{#1}}}
\newcommand{\wgtm}[1]{\ensuremath{\wgtM\at{#1}}}
\newcommand{\wgtt}[1]{\ensuremath{\wgtT\at{#1}}}
\newcommand{\wgtc}[1]{\ensuremath{\wgtC\at{#1}}}
%
%   local notation
%
 %   m(x)
\newcommand{\pw}[1]{\ensuremath{\mu\at{#1}}}           
%   m_T     total choices 
\newcommand{\pwT}{\ensuremath{\mu_{\TCHOICEset}}}   
%   m_T(x)
\newcommand{\pwt}[1]{\ensuremath{\pwT\at{#1}}}     
%   m_M     stable models  
\newcommand{\pwM}{\ensuremath{\mu_{\MODELset}}}   
%   m_M(x)
\newcommand{\pwm}[1]{\ensuremath{\pwM\at{#1}}}     
%   m_C     eq. classes 
\newcommand{\pwC}{\ensuremath{\mu_{\textrm{\cla{C}}}}}   
%   m_C(x)
\newcommand{\pwc}[1]{\ensuremath{\pwC\at{#1}}}     
%   m_E     events 
\newcommand{\pwE}{\ensuremath{\mu_{\EVENTSset}}}   
%   m_E(x)
\newcommand{\pwe}[1]{\ensuremath{\pwE\at{#1}}}     
%
%
%
\newcommand{\stablecore}[1]{\ensuremath{\left\llbracket #1 \right\rrbracket}}
\newcommand{\inconsistent}{\bot}
\newcommand{\given}{\ensuremath{~\middle|~}}
\newcommand{\consequenceclass}{\ensuremath{\Lambda}}
\newcommand{\indepclass}{\ensuremath{\Diamond}}
\newcommand{\probfact}[2]{\ensuremath{#1:#2}}
\newcommand{\probrule}[3]{\probfact{#1}{#2} \leftarrow #3}
\newcommand{\weightfact}[2]{\ensuremath{#1:#2}}
\newcommand{\weightrule}[3]{\weightfact{#1}{#2} \leftarrow #3}
\newcommand{\class}[1]{\ensuremath{[{#1}]_{\sim}}}
\newcommand{\tcgen}[1]{\MODELset\at{#1}}
\newcommand{\condsymb}[2]{\ensuremath{p_{#1|#2}}}
\newcommand{\lpmln}{\texttt{LP\textsuperscript{MLN}}}
\newcommand{\emptyevent}{\ensuremath{\lambda}}
% \newcommand{\powerset}[1]{\ensuremath{\mathbb{P}\at{#1}}}
\newcommand{\powerset}[1]{\ensuremath{\mathrm{2}^{#1}}}
%
%   Acronyms
%
\acrodef{BK}[BK]{background knowledge}
\acrodef{ASP}[ASP]{answer set program}
% \acrodef{ASP}[ASP]{answer set prolog}
\acrodef{NP}[NP]{normal program}
\acrodef{DP}[DP]{disjunctive program}
\acrodef{LP}[LP]{logic program}
\acrodef{PCR}[PCR]{program with choice rules}
\acrodef{DS}[DS]{distribution semantics}
\acrodef{PF}[PF]{probabilistic fact}
\acrodef{WF}[WF]{weighted fact}
\acrodef{TC}[TC]{total choice}
\acrodef{SM}[SM]{stable model}
\acrodef{SC}[SC]{stable core}
\acrodef{KL}[KL]{Kullback-Leibler}
\acrodef{SBF}[SBF]{simple but fruitful}
\acrodef{RSL}[RSL]{random set of literals}
\acrodef{RCE}[RCE]{random consistent event}
\acrodef{SASP}[SASP]{stochastic answer set program}
\acrodef{WASP}[WASP]{weighted answer set program}
\acrodef{HMM}[HMM]{hidden Markov model}
\acrodef{MAP}[MAP]{maximum a posteriori}
\acrodef{MLE}[MLE]{maximum likelihood estimation}
\acrodef{BI}[BI]{Bayesian inference}
\acrodef{PLP}[PLP]{probabilistic logic programming}
%
%   Common objects
%
\newcommand{\sbf}{\ensuremath{\mathrm{1}}}
%
%   Reviewing
%
\newcounter{revcounter}
\newcommand{\LOOK}{\ensuremath{\blacksquare}}
\newcommand{\delete}[1]{\sout{#1}}
\newcommand{\sidenote}[1]{\stepcounter{revcounter}{\color{red!50!black}\(\vert^{\arabic{revcounter}}\)}\marginpar{{\color{red!50!black}\(^{\arabic{revcounter}}\vert\)}\scriptsize #1}}
\newcommand{\defnote}[1]{\marginpar{\scriptsize{{\color{blue!50!black}\bf def.~}#1}}}
\newcommand{\topicnote}[1]{{\scriptsize\color{red!50!black}$\blacktriangleright$}\marginpar{\scriptsize{\color{red!50!black}\it #1}}}
\newcommand{\replace}[2]{\delete{#1}\sidenote{#2}}
\newcommand{\franc}[1]{{\color{green!30!black}#1}}
\newcommand{\bruno}{\color{red!60!black}}
%\newcommand{\spa}[1]{{\color{brown!80!black}{#1}}}
\newcommand{\dietmar}[1]{{\color{brown!40!black}#1}}

% -- side notes --

\newcommand{\selfnote}[1]{\todo[backgroundcolor=green!20]{{\footnotesize #1}}}
\newcommand{\spanote}[1]{{\todo[size=footnotesize,color=teal!20]{\textbf{SPA:} #1}}}
\newcommand{\dsnote}[1]{{\todo[size=footnotesize,color=teal!20]{\textbf{DS:} #1}}}
\newcommand{\francnote}[1]{{\todo[size=footnotesize,color=green!30]{\textbf{FC:} #1}}}
\newcommand{\bdnote}[1]{{\todo[size=footnotesize,color=red!60]{\textbf{BD:} #1}}}

\usepackage{geometry}
\geometry{
    paper=a4paper,
    left=1cm,
    right=5cm,
    top=1cm,
    bottom=2cm
}

\title{Weighted Answer Set Programs}
\author{Francisco, Bruno, Salvador, Dietmar}

\begin{document}

\maketitle

\begin{abstract}
    Drawing inspiration from HMMs; State of the art of PLP and Probabilistic ASP; Leveraging current Prob ASP systems; 
\end{abstract}


\noindent Using a \acl{LP} to model and reason over a real world scenario is often hard because of uncertainty underlying the problem being worked on.
Classic \aclp{LP} represent knowledge in precise and complete terms, which turns out to be problematic when the scenario is characterized by stochastic or observability factors. For example, medical exams illustrate both problems: a system with unreachable parts \eg\ some parts of a living organism can't be directly observed; a sensor that adds noise to real values \eg\ limits and imperfections of instrumentation.

In this work we aim to explore how \aclp{ASP} plus weight augmented facts can lead to useful characterizations for this class of problems; We assume that knowledge about a \emph{system} includes both a theoretical \emph{model}\sidenote{Use `answer set' instead of `stable model'?} and empirical \emph{data} such that:
\begin{itemize}\tight
    \item The model is an \acl{ASP} whose \aclp{SM} are the the system states.
    \item The data is a set of observations and each observation is a set of literals from the model.
    \item The weights in the augmented facts are propagated to the \aclp{SM} and, in general, to observations such as above.
\end{itemize} 

In this setting the empirical distribution from the data can be used in two different tasks:
\begin{enumerate}\tight
    \item Estimate parameteres required in the propagation process above.
    \item Evaluate the model \emph{wrt.}\ the observations.
\end{enumerate}

To frame this setting  we assume that the atoms in the model are associated to sensors of the system's states. 
More specifically, if $a$ an atom, a state can activate the sensor $a$ or $\neg a$ (strong negation) whereas no activation is represented by $\naf a$ and $\naf \neg a$ (default negation).
This redundancy is required to model hidden parts of a system as well as faulty sensors.
\sidenote{Look carefully to the relation between missing values in observations and default negation.}
For example, in
$$\left\{ a, \neg a, \naf b, \naf \neg b, \naf c, \neg c \right\}$$
both $a$ and $\neg a$ are activated (suggesting a fault in the relevant sensors), $b$ is not observed (hidden?), and $\naf c, \neg c$ reports the (consistent) activation of $\neg c$ and no activation of $c$.
If we omit the default negations and use $\co{x}$ to denote $\neg x$ this observation can be shortened to the equivalent form $a\co{a}\co{c}$. 

\begin{example}[Coins, Cards, and Dices]
    Consider a scenario where a coin is tossed and, if it lands in \emph{heads}, $a$, then a dice is thrown, $b$, or a card is drawn, $c$. If the coin lands in \emph{tails}, $\neg a$, no more action is performed. A model of this is, for example, the program  
    \begin{equation}\label{eq:derived.fruitful}
        P_{\mathrm{ccd}} = \left\lbrace\begin{aligned}
            a \vee \neg{a}  & ,          %
            \\
            b \vee c        & \clause a
        \end{aligned}
        \right.
    \end{equation}
    that has atoms $\set{a, b, c}$, literals $\set{a, \neg a, b, \neg b, c, \neg c}$ and, using the shortned form, \aclp{SM}
    \begin{equation}
        ab, ac, \co{a}.
    \end{equation}
    
    
    Possible observations include:
    \begin{itemize}\tight
        \item $ac$, short of $\{a,\naf\neg a,\naf b, \naf \neg b, c, \naf \neg c\}$: a \acl{SM}, therefore a system state, observed \emph{heads} and \emph{card}.
        \item $a$: not a \acs{SM} but contained in the two \acsp{SM} $ab$ and $ac$; observed \emph{heads} but we have no information about the \emph{cards} or the \emph{dice}.
        \item $\co{a}\co{b}$: not a \acs{SM} but contains the \acs{SM} $\co{a}$; coin observed as \emph{not heads} (\emph{tails}?); also we recorded no card drawn but have no information about the dice.
        \item $b,c$: not related with any \acs{SM}.
        \item $a\co{a}$: an inconsistent observation.
    \end{itemize}    
\end{example}

We use sets of literals, instead of atoms, to represent observations because they make possible the distintion between a \emph{explicit negative observation} $\neg\alpha$ (\eg ``I see tails.'') and \emph{not observed} $\naf \alpha$ (\eg ``The coin in hidden.''). This corresponds to assuming a ``boolean sensor'' for each $\alpha$ and $\neg \alpha$.

So, the program \ref{eq:derived.fruitful} defines the ``sensors''
\begin{equation*}
    a, \neg a, b, \neg b, c, \neg c
\end{equation*}

\nocite{*}

\printbibliography
\end{document}