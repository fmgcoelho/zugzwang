\usepackage{multirow}

% LNAI stuff

% TLP stuff

\usepackage{times}
% \usepackage{soul} % to strikeout, highline, etc.
\usepackage{url}
\usepackage[hidelinks]{hyperref}
\usepackage{graphicx}
\usepackage{booktabs}
\usepackage{algorithm}
\usepackage{algorithmic}
\urlstyle{same}
\usepackage[%textsize=tiny,
    prependcaption]{todonotes}
%

% OUR PACKAGES AND MACROS


\usepackage{tikz}
\tikzset{
    boxed/.style={draw, rectangle, rounded corners=4pt, fill=gray!10,inner sep=1em, anchor=center},
    event/.style={},
    smodel/.style={fill=gray!25},
    tchoice/.style={draw, circle},
    indep/.style={},
    proptc/.style = {-latex, dashed},
    propsm/.style = {-latex, thick},
    doubt/.style = {gray} }
\usetikzlibrary{calc, positioning, patterns, perspective}
%
\usepackage{hyperref}
\hypersetup{
    colorlinks=true,
    linkcolor=blue,
    citecolor=blue,
    urlcolor=blue, }
%
\usepackage{commath}
\newtheorem{assumption}{Assumption}
\usepackage{amssymb}
\usepackage[normalem]{ulem}
\usepackage[euler-digits,euler-hat-accent]{eulervm}
\usepackage[nice]{nicefrac}
\usepackage{stmaryrd}
\usepackage{acronym}
\usepackage{multicol}
\usepackage{multirow}
\usepackage{cleveref}
\crefname{example}{ex.}{exs.}
\crefname{proposition}{prop.}{props.}
\crefname{assumption}{asp.}{asps.}
\usepackage{hyphenat}
\hyphenation{
mo-dels mi-ti-ga-ted mi-ni-mal ex-am-ple spe-ci-fi-ca-tion
}

% Environments

\newtheorem{example}{Example}
\newtheorem{definition}{Definition}
\newtheorem{proposition}{Proposition}

% Commands

\def\tight{%
  \itemsep 0pt plus 1pt
  \parskip 0pt plus 1pt}

\newcommand{\ie}{\emph{i.e.}}
\newcommand{\eg}{\emph{e.g.}}
\newcommand{\eat}[1]{}
\newcommand{\at}[1]{\ensuremath{\!\del{#1}}}        %   argument
%
%   Notation
%
%   special sets
%
\newcommand{\cla}[1]{\ensuremath{{\cal #1}}}        %   class of something  
\newcommand{\clx}[1]{\ensuremath{{\mathbb{#1}}}}
%
%   connectives in logic programs
%
\newcommand{\conj}{\ensuremath{\wedge}} \newcommand{\disj}{\ensuremath{\vee}}
\newcommand{\clause}{\ensuremath{\leftarrow}}
\DeclareMathOperator{\naf}{\sim\!}
% \newcommand{\naf}{\ensuremath{\sim\!\!}}
\newcommand{\co}[1]{\ensuremath{\overline{#1}}}     %   complement
\renewcommand{\complement}{\ensuremath{\mathcal{C}}}     %   complement
%
%   often used sets
%
\newcommand{\ATOMSset}{\ensuremath{\cla{A}}}
\newcommand{\LITERALSset}{\ensuremath{\cla{L}}}
\newcommand{\PATOMset}{\ensuremath{\ATOMSset_{\cla{P}}}}
\newcommand{\FACTSset}{\ensuremath{\cla{F}}}
\newcommand{\PROBFset}{\ensuremath{\cla{P}}}
\newcommand{\RULESset}{\ensuremath{\cla{R}}}
\newcommand{\TCHOICEset}{\ensuremath{\cla{T}}}
\newcommand{\MODELset}{\ensuremath{\cla{M}}}
\newcommand{\EVENTSset}{\ensuremath{\cla{E}}}
\newcommand{\CONSISTset}{\ensuremath{\cla{W}}}
%
%   often used functions
%
%   error (loss)
\newcommand{\err}[1]{\ensuremath{\mathrm{err}\at{#1}}}
%
%   probability 
\newcommand{\prfunc}{\ensuremath{\mathrm{P}}}       
%   P(X = x)
\newcommand{\pr}[1]{\ensuremath{\prfunc\at{#1}}}    
%   P_X(x)
\newcommand{\prd}[1]{\ensuremath{\prfunc_{#1}}}     
\newcommand{\prT}{\prd{\TCHOICEset}}
\newcommand{\prM}{\prd{\MODELset}}
\newcommand{\prE}{\prd{\EVENTSset}}
\newcommand{\prC}{\prd{\cla{C}}}
%
%   local notation
%
 %   m(x)
\newcommand{\pw}[1]{\ensuremath{\mu\at{#1}}}           
%   m_T     total choices 
\newcommand{\pwT}{\ensuremath{\mu_{\TCHOICEset}}}   
%   m_T(x)
\newcommand{\pwt}[1]{\ensuremath{\pwT\at{#1}}}     
%   m_M     stable models  
\newcommand{\pwM}{\ensuremath{\mu_{\MODELset}}}   
%   m_M(x)
\newcommand{\pwm}[1]{\ensuremath{\pwM\at{#1}}}     
%   m_C     eq. classes 
\newcommand{\pwC}{\ensuremath{\mu_{\textrm{\cla{C}}}}}   
%   m_C(x)
\newcommand{\pwc}[1]{\ensuremath{\pwC\at{#1}}}     
%   m_E     events 
\newcommand{\pwE}{\ensuremath{\mu_{\EVENTSset}}}   
%   m_E(x)
\newcommand{\pwe}[1]{\ensuremath{\pwE\at{#1}}}     
%
%
%
\newcommand{\stablecore}[1]{\ensuremath{\left\llbracket #1 \right\rrbracket}}
\newcommand{\inconsistent}{\bot}
\newcommand{\given}{\ensuremath{~\middle|~}}
\newcommand{\consequenceclass}{\ensuremath{\Lambda}}
\newcommand{\indepclass}{\ensuremath{\Diamond}}
\newcommand{\probfact}[2]{\ensuremath{#1:#2}}
\newcommand{\probrule}[3]{\probfact{#1}{#2} \leftarrow #3}
\newcommand{\class}[1]{\ensuremath{[{#1}]_{\sim}}}
\newcommand{\tcgen}[1]{\MODELset\at{#1}}
\newcommand{\condsymb}[2]{\ensuremath{p_{#1|#2}}}
\newcommand{\lpmln}{\texttt{LP\textsuperscript{MLN}}}
\newcommand{\emptyevent}{\ensuremath{\lambda}}
\newcommand{\powerset}[1]{\ensuremath{\mathbb{P}\at{#1}}}
%
%   Acronyms
%
\acrodef{BK}[BK]{background knowledge}
\acrodef{ASP}[ASP]{answer set program}
\acrodef{NP}[NP]{normal program}
\acrodef{LP}[LP]{logic program}
\acrodef{PCR}[PCR]{program with choice rules}
\acrodef{DS}[DS]{distribution semantics}
\acrodef{PF}[PF]{probabilistic fact}
\acrodef{TC}[TC]{total choice}
\acrodef{SM}[SM]{stable model}
\acrodef{SC}[SC]{stable core}
\acrodef{KL}[KL]{Kullback-Leibler}
\acrodef{SBF}[SBF]{simple but fruitful}
\acrodef{RSL}[RSL]{random set of literals}
\acrodef{RCE}[RCE]{random consistent event}
\acrodef{SASP}[SASP]{stochastic answer set program}
\acrodef{WASP}[WASP]{weighted answer set program}
\acrodef{HMM}[HMM]{hidden Markov model}
%
%   Common objects
%
\newcommand{\sbf}{\ensuremath{\mathrm{sbf}}}
%
%   Reviewing
%
\newcounter{revcounter}
\newcommand{\LOOK}{\ensuremath{\blacksquare}}
\newcommand{\delete}[1]{\sout{#1}}
\newcommand{\sidenote}[1]{\stepcounter{revcounter}{\color{red!50!black}\(\vert^{\arabic{revcounter}}\)}\marginpar{{\color{red!50!black}\(^{\arabic{revcounter}}\vert\)}\footnotesize #1}}
\newcommand{\replace}[2]{\delete{#1}\sidenote{#2}}
%\newcommand{\franc}[1]{{\color{green!30!black}#1}}
\newcommand{\bruno}{\color{red!60!black}}
%\newcommand{\spa}[1]{{\color{brown!80!black}{#1}}}
\newcommand{\dietmar}[1]{{\color{brown!40!black}#1}}

% -- side notes --

\newcommand{\selfnote}[1]{\todo[backgroundcolor=green!20]{{\footnotesize #1}}}
\newcommand{\spa}[1]{{\todo[size=footnotesize,color=teal!20]{\textbf{SPA:} #1}}}
\newcommand{\dsnote}[1]{{\todo[size=footnotesize,color=teal!20]{\textbf{DS:} #1}}}
\newcommand{\franc}[1]{{\todo[size=footnotesize,color=green!30]{\textbf{FC:} #1}}}
\newcommand{\bdnote}[1]{{\todo[size=footnotesize,color=red!60]{\textbf{BD:} #1}}}
