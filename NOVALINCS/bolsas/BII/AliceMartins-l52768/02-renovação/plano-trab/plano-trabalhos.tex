% !TeX program = xelatex
\documentclass{oficio}

\usepackage{polyglossia}
\setmainlanguage{portuges}

\def\ASSINATURA{\includegraphics{fcsig.pdf}}

\begin{document}
\unidade{\textsc{\DI}}
\documento{%
NL/ZZ/BII/1/AM/2
}{%
Renovação de BII --- Plano de Trabalhos}{%
31 de julho de}
%\EU{}

\qualidade{(Professor Auxiliar)}

Considerando que foi já implementada uma biblioteca \texttt{Python} que proporciona as seguintes funcionalidades:
%
\begin{itemize}
    \item Ler a descrição de uma rede Bayesiana num formato comum (\texttt{BIF}) para uma representação intermédia.
    \item Escrever uma especificação \texttt{ASP} anotada com probabilidades a partir da representação intermédia indicada no ponto anterior.
\end{itemize}
%
e que, com esta biblioteca é possível aplicar métodos e ferramentas desenvolvidos no âmbito do projeto ``\emph{Zugzwang | Lógica e Inteligência Artificial}'' a um conjunto alargado de problemas, \textbf{pretende-se continuar,} com a implementação duma biblioteca \texttt{Python} para processar ficheiros de código ASP anotados com probabilidades incluindo as seguintes funcionalidades:

\begin{itemize}

    \item Extração das anotações e associação aos factos relevantes.
    \item Comunicação com o sistema Potassco para obtenção dos resultados do programa ASP subjacente.
    \item Cálculos de probabilidades com base nos modelos estáveis obtidos no passo anterior, nas anotações extraídas no passo 1, e nos resultados teóricos deste projeto.    
\end{itemize}

\CHAU{Com os melhores cumprimentos,}

\end{document}